\documentclass[12pt]{article}

\topmargin -40pt
\marginparwidth 0pt
\oddsidemargin  -40pt
\evensidemargin 0pt
\marginparsep 0pt
\textwidth 7.2 in
\textheight  10 in
\hoffset  0.1in

\usepackage{amsthm,amsmath,amssymb,amscd,verbatim,epsfig}
\usepackage{mathptmx}
\usepackage{amsfonts}
%\usepackage{setapace}
\usepackage{graphicx}
\usepackage{bm}
%\usepackage{CJK}
\usepackage{ulem}
\usepackage{multicol}
\usepackage{enumerate}
\usepackage{float}
\usepackage{fontspec}
\usepackage{xeCJK}
\setmainfont{Times New Roman}
\setCJKmainfont{TaipeiSansTCBeta-Regular}
\XeTeXlinebreaklocale "zh"
\XeTeXlinebreakskip = 0pt plus 1pt

\title{Homework 2 of Computational Mathematics}
\author{AM15 黃琦翔 111652028}

\begin{document}
\maketitle
\begin{enumerate}
    \item $x^3 = x + 1\implies x^2 = 1 + \dfrac{1}{x}\implies x = \sqrt{1 + \dfrac{1}{x}} = g(x)$.
    $p_1 = g(p_0) = \sqrt{1 + 1} = \sqrt{2} \approx 1.414$.
    $p_2 = g(p_1) = \sqrt{1 + \dfrac{1}{\sqrt{2}}} \approx 1.3065$.
    $p_3 = g(p_2) \approx 1.3172$.
    $p_4 = g(p_3) \approx 1.326$.
    $p_5 = g(p_4) \approx 1.324$ 
    Then, $p_4$ is the answer that we want to find.

    \item Let $a_0 = 1, b_0 = 4$.
    $f(a_0) = -2$, $f(b_0) = $
    $f(2.5) = 14.125$
\end{enumerate}
\end{document}
