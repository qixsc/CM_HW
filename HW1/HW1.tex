\documentclass[12pt]{article}

\topmargin -40pt
\marginparwidth 0pt
\oddsidemargin  -40pt
\evensidemargin 0pt
\marginparsep 0pt
\textwidth 7.2 in
\textheight  10 in
\hoffset  0.1in

\usepackage{amsthm,amsmath,amssymb,amscd,verbatim,epsfig}
\usepackage{mathptmx}
\usepackage{amsfonts}
%\usepackage{setapace}
\usepackage{graphicx}
\usepackage{bm}
%\usepackage{CJK}
\usepackage{ulem}
\usepackage{multicol}
\usepackage{enumerate}
\usepackage{float}
\usepackage{fontspec}
\usepackage{xeCJK}
\setmainfont{Times New Roman}
\setCJKmainfont{TaipeiSansTCBeta-Regular}
\XeTeXlinebreaklocale "zh"
\XeTeXlinebreakskip = 0pt plus 1pt

\title{Homework 1 of Computational Mathematics}
\author{AM15 黃琦翔 111652028}

\begin{document}
\maketitle
\begin{enumerate}
    \item $f(x) = x^3 + 2x + k$, then $f'(x) = 3x^2 + 2 > 0$ for all $x$.
    Thus, we assume there are two points $a, b \in \mathbb{R}$ s.t. $f(a) = f(b) = 0$.
    By Rolle's Theorem, there exists a point $c$ in $[a, b]$(or $[b, a]$) s.t. $f'(c) = 0$(Contradiction).

    And since $f(x) \to \infty$ as $x \to \infty$ and $f(x) \to -\infty$ as $x\to -\infty$,
    by IVT, there eists at least one $x$ s.t. $f(x) = 0$.
    Thus, the graph of $f(x)$ crosses the $x$-axis exactly once whatever $k$ is.

    \item By EVT, we know that the maximun occurs either $f'(x) = 0$ or $a, b$.\begin{enumerate}
        \item $f'(x) = \dfrac{1}{3}(2 - e^{x}) = 0$ when $x = \ln(2)$.
        And since $f'(x) > 0$ when $x \in (0, \ln(2))$ and $f'(x) < 0$ when $x\in (\ln(2), 1)$,
        $f(\ln(2)) = \dfrac{1}{3}(2-2+2\ln2) = \dfrac{2\ln(2)}{3}$ is the maximun.

        \item $f'(x) = \dfrac{4x^2 - 8x - (4x-3)(2x-2)}{x^4 - 4x^3 + 4x^2} = \dfrac{-4x^2 + 6x - 8}{x^4 - 4x^3 + 4x^2} < 0$ for $x \in [0.5, 1]$.
        Thus, $f(0.5) = \dfrac{2-3}{0.25-1} = \dfrac{4}{3}$.

        \item $f'(x) = 2\cos(2x) - 4x\sin(2x) - 2x + 4 = 0$, $x \approx 3.1311062779876$ and $f(x) \approx 4.981433957553$.
        And $|f(2)| \approx 2.61457$, $|f(4)| \approx 37.1640$.
        Then, $\max |f(x)| \approx 37.1640$

        \item $f'(x) = \sin(x-1)e^{-\cos(x-1)}$ since $\sin(x) > 0$ for $0 < x < 1$ and $e^{x}$ is always positive, 
        the maximun is $f(2) = 1 + e^{-\cos(1)}$.
    \end{enumerate}

        \item $f'(x) = e^x(\cos(x) - \sin(x))$, $f''(x) = -2e^x\sin(x)$, and $f^{(3)}(x) = -2e^x(\sin(x) + \cos(x))$. 
        Then, $P_2(x) = f(0) + f'(0)x + \dfrac{1}{2} f''(0)x^2 = 1 + x$ and $R_2 = \dfrac{f^{(3)}(\xi(x))}{6}x^3 = \dfrac{-1}{3}x^3e^{\xi(x)}(\sin(\xi(x)) + \cos(\xi(x)))$.
        Thus, we have $f(x) = e^x\cos(x) = 1 + x - \dfrac{1}{3}x^3e^{\xi(x)}(\sin(\xi(x)) + \cos(\xi(x)))$
    \begin{enumerate}
        \item $P_2(\dfrac{1}{2}) = 1 + \dfrac{1}{2} = \dfrac{3}{2}$. And \begin{align*}
            |f(\dfrac{1}{2}) - P_2(\dfrac{1}{2})| &= |R_2(\dfrac{1}{2})|\\
            &= \dfrac{1}{3} \dfrac{1}{2^3} e^{\xi(\frac{1}{2})}(\sin(\xi(\dfrac{1}{2})) + \cos(\xi(\dfrac{1}{2})))\\
                &\leq \dfrac{1}{24} e^{\frac{1}{2}}(\sin(\dfrac{1}{2}) + \sin(\dfrac{1}{2}))\\
                &\approx  0.09322200499
        \end{align*}
        And the actual error is $0.05311086942$.

        \item The bound is the maximun of $|R_2(x)|$ for $x \in [0, 1]$.
        Thus, the bound$ = |R_2(x)| \leq \dfrac{1}{3} 1^3 e(\sqrt{2}) = 1.28141034272$.

        \item $\displaystyle\int_{0}^{1} P_2(x)\ dx = \displaystyle\int_{0}^{1} 1 + x\ dx = 1 + \dfrac{1}{2} = \dfrac{3}{2}$.
        
        \item \begin{align*}
            \int_{0}^{1} R_2(x)\ dx| &=\int_{0}^{1} \dfrac{1}{3}x^3 e^{\xi(x)}(\sin(x) + \cos(x))\ dx\\
            &\leq \int_{0}^{\frac{\pi}{4}} \dfrac{1}{3}x^3 e^x (\sin(x) + \cos(x))\ dx+ \int_{\frac{\pi}{4}}^{1} \dfrac{\sqrt{2}}{3}x^3e^x\ dx\\
            &\approx 0.08328 + 0.1809\\
            &= 0.26418
        \end{align*}
        And the actual error is $1.5 - 1.3780 = 0.122$.
    \end{enumerate}

    \item $f(x) = \dfrac{1}{1-x}$, $P_n(x) = \displaystyle\sum_{k=0}^{n} x^k$.
    Then, the remainder is $\dfrac{n!}{n! \cdot (1-\xi(x))^{n+1}}x^{n+1} < \dfrac{x^{n+1}}{1-x}$.
    Thus, we only need to find the minimun $n$ s.t. $\dfrac{0.5^{n+1}}{1-0.5} = 0.5^n < 10^{-6}$.
    By taking log both side, $n(-0.30102999566) < -6\implies n \geq 20$.
    Thus, $n = 20$.

    \item Since the relative error is $\left|\dfrac{p^*-p}{p}\right| \leq 10^{-4}$, $p - p\times 10^{-4} \leq p^* \leq p + p\times 10^{-4}$.
    \begin{enumerate}
        \item $[\pi - \pi\times 10^{-4}, \pi + \pi\times 10^{-4}]$
        \item $[e - e\times 10^{-4}, e + e\times 10^{-4}]$
        \item $[\sqrt{2} - \sqrt{2}\times 10^{-4}, \sqrt{2} + \sqrt{2}\times 10^{-4}]$
        \item $[\sqrt[3]{7} - \sqrt[3]{7}\times 10^{-4}, \sqrt[3]{7} + \sqrt[3]{7}\times 10^{-4}]$
    \end{enumerate}

    \item \begin{enumerate}
        \item Sicne $e^{0} - e^{-0} = 0$, $\displaystyle\lim_{x \to 0} \dfrac{e^{x} - e^{-x}}{x} \overset{\text{L'H}}{=} \displaystyle\lim_{x\to 0} \dfrac{2e^{x}}{1} = 2$.
        \item $f(0.1) = \dfrac{0.111\times 10^1 - 0.905}{0.1} = 0.205\times 10^2 = .205 \times 10^1$.
        \item $M_{3}(x) = 1 + x + \dfrac{x^2}{2} + \dfrac{x^3}{6} = ((\dfrac{1}{6}x + \dfrac{1}{2})x + 1)x + 1$ 

        , then $f(x) = \dfrac{(1 + x + \frac{1}{2}x^2 + \frac{1}{6}x^3) - (1 - x + \frac{1}{2}x^2 - \frac{1}{6}x^3)}{x} = \dfrac{2x + \frac{1}{3}x^3}{x} = 2 + \dfrac{1}{3}x^2$.
        Thus, $f(0.1) = 2 + .333 * .01= 2 + .333e-2 = 2$
    
        \item The relative error of (b) $= \dfrac{2.05 - f(.1)}{f(.1)} = .232937 \times 10^{-1}$.
        And the relative error of (c) $= \dfrac{2 - f(.1)}{f(.1)} = .166 \times 10^{-2}$.
    \end{enumerate}

    \item \begin{enumerate}
        \item $= (-1)^0 \times (2^{1024+8+2 - 1023}) \times (1 + 2^{-1} + 2^{-4} + 2^{-7} + 2^{-8}) = 2^11 + 2^10 + 2^7 + 2^4 + 2^3\\
        = 2048+1024+128+16+8 = 3224 = .3224 \times 10^{4}$.
        \item Observe it and we get it is $-1\times $ the number of (a), then it is $-.3224 \times 10^{4}$.
        \item $= (2^{1024-1 - 1023}) \times (1 + 2^{-2} + 2^{-4} + 2^{-7} + 2^{-8}) = .132421875 \times 10^1$.
        \item $= .132421875 \times 10^1 + 2^{-52}$.
    \end{enumerate}

    \item \begin{enumerate}
        \item \begin{align*}
            m &= \dfrac{.1013}{.1130} = .8965\\
            d_1 &= -.6099e1 + .8965\times .6990e1 = .168\\
            f_1 &= .1422e2 - .8965 \times .1420e2 = .1490e1\\
            y &= \dfrac{.1490e1}{.168} = .8869e1\\
            x &= \dfrac{.1420e2 + .6990e1 \times .8969e1}{.1130e1} = .6742e2
        \end{align*}

        \item \begin{align*}
            m &= \dfrac{-.1811e2}{.8110e1} = -.2233e1\\
            d_1 &= .1122e3 + .2233e1 \times .1220e2 = .1394e3\\
            f_1 &= -.1376 - .2233e1 \times .1370 = -.4435\\
            y &= \dfrac{-.4435}{.1394e3} = -.3181e-2\\
            x &= \dfrac{-.1370 + .1220e2 \times .3181e-2}{.8110e1} = -.1211e-1\\
        \end{align*}
    \end{enumerate}

    \item The upper bound of length is $3.5,\ 4.5,\ 5.5$ and the lower bound of length is $2.5,\ 3.5,\ 4.5$.
    Thus, the upper bound of volume is $3.5 \cdot 4.5 \cdot 5.5 = .83e2$, the lower bound of volume is $.39e2$.
    The upper bound of surface area is $32 + 38 + 50 = .12e3$, the lower bound of surface area is $17 + 22 + 32 = .71e2$.

    \item \begin{enumerate}
        \item $\sin\left(\dfrac{1}{n^2}\right) = 0 + \dfrac{1}{n^2} - \dfrac{1}{6}\left(\dfrac{1}{n^2}\right)^3 + \cdots = 0 + O\left(\dfrac{1}{n^2}\right)$.

        \item $(\sin\left(\dfrac{1}{n}\right))^2 = (0 + O\left(\dfrac{1}{n}\right))^2 = O\left(\dfrac{1}{n^2}\right)$.
        
        \item $\dfrac{\sin(h)}{h} = 1 +  \dfrac{1}{2}\cdot (-\dfrac{1}{3}) h^2 + \cdots = O(h^2)$.
        
        \item $\dfrac{1-e^h}{h} = -1 + (-\dfrac{1}{2})h + \cdots = O(h)$.
    \end{enumerate}

    \item\begin{enumerate}
        \item $F(x) = c_1F_1(x) + c_2F_2(x) = c_1L_1 + O(x^\alpha) + c_2L_2 + O(x^\beta) = c_1L_1 + c_2L_2 + O^{x^{\gamma}}$.
        
        \item $G(x) = F_1(c_1x) + F_2(c_2x) = L_1 + O(x^\alpha) + L_2 + O(x^\beta) = L_1 + L_2 + O(x^\gamma)$.
    \end{enumerate}
\end{enumerate}
\end{document}
