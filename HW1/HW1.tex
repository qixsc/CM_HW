\documentclass[12pt]{article}

\topmargin -40pt
\marginparwidth 0pt
\oddsidemargin  -40pt
\evensidemargin 0pt
\marginparsep 0pt
\textwidth 7.2 in
\textheight  10 in
\hoffset  0.1in

\usepackage{amsthm,amsmath,amssymb,amscd,verbatim,epsfig}
\usepackage{mathptmx}
\usepackage{amsfonts}
%\usepackage{setapace}
\usepackage{graphicx}
\usepackage{bm}
%\usepackage{CJK}
\usepackage{ulem}
\usepackage{multicol}
\usepackage{enumerate}
\usepackage{float}
\usepackage{fontspec}
\usepackage{xeCJK}
\setmainfont{Times New Roman}
\setCJKmainfont{TaipeiSansTCBeta-Regular}
\XeTeXlinebreaklocale "zh"
\XeTeXlinebreakskip = 0pt plus 1pt

\title{Homework 1 of Computational Mathematics}
\author{AM15 黃琦翔 111652028}

\begin{document}
\maketitle
\begin{enumerate}
    \item $f(x) = x^3 + 2x + k$, then $f'(x) = 3x^2 + 2 > 0$ for all $x$.
    Thus, we assume there are two points $a, b \in \mathbb{R}$ s.t. $f(a) = f(b) = 0$.
    By Rolle's Theorem, there exists a point $c$ in $[a, b]$(or $[b, a]$) s.t. $f'(c) = 0$(Contradiction).

    And since $f(x) \to \infty$ as $x \to \infty$ and $f(x) \to -\infty$ as $x\to -\infty$,
    by IVT, there eists at least one $x$ s.t. $f(x) = 0$.
    Thus, the graph of $f(x)$ crosses the $x$-axis exactly once whatever $k$ is.

    \item By EVT, we know that the maximun occurs either $f'(x) = 0$ or $a, b$.\begin{enumerate}
        \item $f'(x) = \dfrac{1}{3}(2 - e^{x}) = 0$ when $x = \ln(2)$.
        And since $f'(x) > 0$ when $x \in (0, \ln(2))$ and $f'(x) < 0$ when $x\in (\ln(2), 1)$,
        $f(\ln(2)) = \dfrac{1}{3}(2-2+2\ln2) = \dfrac{2\ln(2)}{3}$ is the maximun.

        \item $f'(x) = \dfrac{4x^2 - 8x - (4x-3)(2x-2)}{x^4 - 4x^3 + 4x^2} = \dfrac{-4x^2 + 6x - 8}{x^4 - 4x^3 + 4x^2} < 0$ for $x \in [0.5, 1]$.
        Thus, 
        
        $f(0.5) = \dfrac{2-3}{0.25-2} = \dfrac{4}{7}$.

        \item $f'(x) = 2\cos(2x) - 4x\sin(2x) - 2x + 4 = 0$

        \item $f'(x) = \sin(x-1)e^{-\cos(x-1)}$
    \end{enumerate}
\end{enumerate}
\end{document}
