\documentclass[12pt]{article}

\topmargin -40pt
\marginparwidth 0pt
\oddsidemargin  -40pt
\evensidemargin 0pt
\marginparsep 0pt
\textwidth 7.2 in
\textheight  10 in
\hoffset  0.1in

\usepackage{amsthm,amsmath,amssymb,amscd,verbatim,epsfig}
\usepackage{mathptmx}
\usepackage{amsfonts}
%\usepackage{setapace}
\usepackage{graphicx}
\usepackage{bm}
%\usepackage{CJK}
\usepackage{ulem}
\usepackage{multicol}
\usepackage{enumerate}
\usepackage{float}
\usepackage{fontspec}
\usepackage{xeCJK}
\setmainfont{Times New Roman}
\setCJKmainfont{TaipeiSansTCBeta-Regular}
\XeTeXlinebreaklocale "zh"
\XeTeXlinebreakskip = 0pt plus 1pt

\title{Homework 1 of Computational Mathematics}
\author{AM15 黃琦翔 111652028}

\begin{document}
\maketitle
\begin{enumerate}
    \item $f(x) = x^3 + 2x + k$, then $f'(x) = 3x^2 + 2 > 0$ for all $x$.
    Thus, we assume there are two points $a, b \in \mathbb{R}$ s.t. $f(a) = f(b) = 0$.
    By Rolle's Theorem, there exists a point $c$ in $[a, b]$(or $[b, a]$) s.t. $f'(c) = 0$(Contradiction).

    And since $f(x) \to \infty$ as $x \to \infty$ and $f(x) \to -\infty$ as $x\to -\infty$,
    by IVT, there eists at least one $x$ s.t. $f(x) = 0$.
    Thus, the graph of $f(x)$ crosses the $x$-axis exactly once whatever $k$ is.

    \item By EVT, we know that the maximun occurs either $f'(x) = 0$ or $a, b$.\begin{enumerate}
        \item $f'(x) = \dfrac{1}{3}(2 - e^{x}) = 0$ when $x = \ln(2)$.
        And since $f'(x) > 0$ when $x \in (0, \ln(2))$ and $f'(x) < 0$ when $x\in (\ln(2), 1)$,
        $f(\ln(2)) = \dfrac{1}{3}(2-2+2\ln2) = \dfrac{2\ln(2)}{3}$ is the maximun.

        \item $f'(x) = \dfrac{4x^2 - 8x - (4x-3)(2x-2)}{x^4 - 4x^3 + 4x^2} = \dfrac{-4x^2 + 6x - 8}{x^4 - 4x^3 + 4x^2} < 0$ for $x \in [0.5, 1]$.
        Thus, 
        
        $f(0.5) = \dfrac{2-3}{0.25-1} = \dfrac{4}{3}$.

        \item $f'(x) = 2\cos(2x) - 4x\sin(2x) - 2x + 4 = 0$, $x \approx 3.1311062779876$.
        And then the maximun is about $4.981433957553$.

        \item $f'(x) = \sin(x-1)e^{-\cos(x-1)}$ since $\sin(x) > 0$ for $0 < x < 1$ and $e^{x}$ is always positive, 
        the maximun is $f(2) = 1 + e^{-\cos(1)}$.
    \end{enumerate}

        \item $f'(x) = e^x(\cos(x) - \sin(x))$, $f''(x) = -2e^x\sin(x)$, and $f^{(3)}(x) = -2e^x(\sin(x) + \cos(x))$. 
        Then, $P_2(x) = f(0) + f'(0)x + \dfrac{1}{2} f''(0)x^2 = 1 + x$ and $R_2 = \dfrac{f^{(3)}(\xi(x))}{6}x^3 = \dfrac{-1}{3}x^3e^{\xi(x)}(\sin(\xi(x)) + \cos(\xi(x)))$.
        Thus, we have $f(x) = e^x\cos(x) = 1 + x - \dfrac{1}{3}x^3e^{\xi(x)}(\sin(\xi(x)) + \cos(\xi(x)))$
    \begin{enumerate}
        \item $P_2(\dfrac{1}{2}) = 1 + \dfrac{1}{2} = \dfrac{3}{2}$. And \begin{align*}
            |f(\dfrac{1}{2}) - P_2(\dfrac{1}{2})| &= |R_2(\dfrac{1}{2})|\\
            &= \dfrac{1}{3} \dfrac{1}{2^3} e^{\xi(\frac{1}{2})}(\sin(\xi(\dfrac{1}{2})) + \cos(\xi(\dfrac{1}{2})))\\
                &\leq \dfrac{1}{24} e^{\frac{1}{2}}(\sin(\dfrac{1}{2}) + \sin(\dfrac{1}{2}))\\
                &\approx  0.09322200499
        \end{align*}
        And the actual error is $0.05311086942$.

        \item The bound is the maximun of $|R_2(x)|$ for $x \in [0, 1]$.
        Thus, the bound$ = |R_2(x)| \leq \dfrac{1}{3} 1^3 e(\sqrt{2}) = 1.28141034272$.

        \item $\displaystyle\int_{0}^{1} P_2(x)\ dx = \displaystyle\int_{0}^{1} 1 + x\ dx = 1 + \dfrac{1}{2} = \dfrac{3}{2}$.
        
        \item \begin{align*}
            \int_{0}^{1} R_2(x)\ dx| &=\int_{0}^{1} \dfrac{1}{3}x^3 e^{\xi(x)}(\sin(x) + \cos(x))\ dx\\
            &\leq \int_{0}^{\frac{\pi}{4}} \dfrac{1}{3}x^3 e^x (\sin(x) + \cos(x))\ dx+ \int_{\frac{\pi}{4}}^{1} \dfrac{\sqrt{2}}{3}x^3e^x\ dx\\
            &\approx 0.08328 + 0.1809\\
            &= 0.26418
        \end{align*}
        And the actual error is $1.5 - 1.3780 = 0.122$.
    \end{enumerate}

    \item $f(x) = \dfrac{1}{1-x}$, $P_n(x) = \displaystyle\sum_{k=0}^{n} x^k$.
    Then, the remainder is $\dfrac{n!}{n! \cdot (1-\xi(x))^{n+1}}x^{n+1} < x^{n+1}$.
    Thus, we only need to find the minimun $n$ s.t. $0.5^{n+1} < 10^{-6}$.
    By taking log both side, $(n+1)(-0.30102999566) < -6\implies n \geq 19$.
    Thus, $n = 19$.

    \item 
\end{enumerate}
\end{document}
